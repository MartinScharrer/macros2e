%&pdflatex
% $Id$
\documentclass[12pt]{article}
\usepackage[margin=1in]{geometry}
\usepackage{ydoc}

\usepackage{booktabs}
\usepackage{tabularx}

\title{List of internal \LaTeX2e Macros useful to Package Authors}
\author{Compiled by Martin Scharrer\\\url{martin@scharrer-online.de}}
\date{Version 0.1a -- February 5th 2011}

\makeatletter
\begin{document}
\maketitle
\begin{abstract}
This document lists the internal macros defined by the \LaTeX2e base files which can be also useful to package authors.
A more detailed description can be found in the `source2e' document from which this list was compiled.

This document is not yet complete in content and format and may miss some macros. Several macros are not yet
sorted in the correct category.
\end{abstract}

\tableofcontents

\clearpage

\section{Constants}

\subsection{Number Constants}

Some of the following integer values are defined using \Macro\countdef\relax (for $-1$), \Macro\chardef\relax (for values between 0--255) and \Macro\mathchardef\relax ($>255$).
There are robust and do not expand in an \Macro\edef\relax context. When used on the right side of an assignment the act the 
same way as a count register.  The |\m@ne| is a real count register and can still be modified, but doing so would certainly break various code.
In the terms defined by \emph{The TeXbook}, this integer constants yield \emph{internal integers} rather than \emph{integer denotations}.

The other numbers are defined simply using macros, which will expand as normal.
They were defined to be used to set the font sizes, which explains the non-integer numbers.
Please note that if they are used for an assignment or other numeric context (e.g.\ |\ifnum|)
TeX will keep expanding the following tokens until it will find a non-numeric token, like a space or |\relax|.


\bigskip
\noindent\hbox to \linewidth\bgroup\hss
\begin{tabular}{lrl}
  \toprule
         Macro     & Value & Defined using \\
  \midrule
   \Macro\@ne      &     1 & chardef \\
   \Macro\tw@      &     2 & chardef \\
   \Macro\thr@@    &     3 & chardef \\
   \Macro\sixt@@n  &    16 & chardef \\
   \Macro\@xxxii   &    32 & chardef \\
   \Macro\@cclv    &   255 & chardef \\
   \Macro\@cclvi   &   256 & mathchardef \\
   \Macro\@m       &  1000 & mathchardef \\
   \Macro\@M       & 10000 & mathchardef \\
   \Macro\@Mi      & 10001 & mathchardef \\
   \Macro\@Mii     & 10002 & mathchardef \\
   \Macro\@Miii    & 10003 & mathchardef \\
   \Macro\@miv     & 10004 & mathchardef \\
   \Macro\@MM      & 20000 & mathchardef \\
  \bottomrule
\end{tabular}
\hss
\begin{tabular}{lrl}
  \toprule
         Macro     & Value & Defined using \\
  \midrule
   \Macro\@vpt     &     5 & def \\
   \Macro\@vipt    &     6 & def \\
   \Macro\@viipt   &     7 & def \\
   \Macro\@viiipt  &     8 & def \\
   \Macro\@ixpt    &     9 & def \\
   \Macro\@xpt     &    10 & def \\
   \Macro\@xipt    & 10.95 & def \\
   \Macro\@xiipt   &    12 & def \\
   \Macro\@xivpt   &  14.4 & def \\
   \Macro\@xviipt  & 17.28 & def \\
   \Macro\@xxpt    & 20.74 & def \\
   \Macro\@xxvpt   & 24.88 & def \\
  \\%\midrule
   \Macro\m@ne     &    -1 & countdef \\
  \bottomrule
\end{tabular}
\hss\egroup


\subsection{Dimension Constants}
The following dimension and skip constants are defined using registers. They must not be changed.

\begin{center}
\begin{tabularx}{\linewidth}{lrlX}
  \toprule
    Macro           & Value         & Defined using & Notes \\
  \midrule
    \Macro\p@       & 1pt           & newdimen      & Can be used as a replacement of `|pt|' behind a number due to the resulting multiplication \\
    \Macro\z@       & 0pt           & newdimen      & Can be used both for 0pt and 0 \\
    \Macro\maxdimen & 16383.99999pt & newdimen      & Largest valid dimension \\
  \bottomrule
\end{tabularx}

\begin{tabularx}{\linewidth}{lrlX}
  \toprule
    Macro             & Value                 & Defined using & Notes\\
    \midrule
    \Macro\z@skip     & 0pt plust0pt minus0pt & newskip       & \\
    \Macro\hideskip   & -1000pt plus 1fill    & newskip       & negative but can grow \\
    \Macro\@flushglue & 0pt plus 1fil         & newskip       & \\
  \bottomrule
\end{tabularx}
\end{center}

\subsection{String Constants}
The following macros hold common strings and are expandable.

\bigskip\noindent
\begin{tabularx}{\linewidth}{llX}
  \toprule
    Macro             & Value  & Note\\
  \midrule

   \Macro\space & One space \\
   \Macro\@spaces & Four spaces \\
   \Macro\empty & Empty string  & Commonly used to define empty macros using \Macro\let. \\
   \Macro\@empty & Empty string & Same as above. Used by the \LaTeX{} kernel commands. \\
   \Macro\@backslashchar & ``\texttt{\@backslashchar}'' & Backslash with catcode 12 (other), i.e. simple ASCII backlash usable in e.g.\ \Macro\write.\\
   \Macro\@percentchar   & ``\texttt{\@percentchar}'' & Percent character with catcode 12 (other), i.e. simple ASCII percent usable in e.g.\ \Macro\write.\\
   \Macro\@charlb        & ``\texttt{\@charlb}'' & Left brace with catcode 11 (letter). \\
   \Macro\@charrb        & ``\texttt{\@charrb}'' & Right brace with catcode 11 (letter). \\
   \Macro\@clsextension  & ``\texttt{cls}'' \\
   \Macro\@pkgextension  & ``\texttt{sty}'' \\
  \bottomrule
\end{tabularx}

\subsection{Token Constants}
The following macros are defined using \Macro\let!\@backslashchar!<macro>'='<token> and therefore equal to this token. They are useful for |\ifx| comparisons
with tokens read by |\let| or |\futurelet|.

\bigskip\noindent
\begin{tabular}{llll}
 \toprule
   Name & Token & Catcode & Note \\
 \midrule
   \Macro\@sptoken & Space & 10 & Should not be confused with \Macro\space \\ 
   \Macro\bgroup   & |{|   &  1 & Begin of group \\
   \Macro\egroup   & |}|   &  2 & End of group \\
 \bottomrule
\end{tabular}

\subsection{Other}

\begin{tabularx}{\linewidth}{lllX}
  \toprule
    Macro            & Value   & Defined using & Notes \\
  \midrule
   \Macro\void@box   & (void)  & newbox        & permanently void box register \\
   \Macro\@undefined & (undef) & (undefined)   & This macro is not defined.
       It is used to test if other macros are undefined (using \Macro\ifx) or set them to an undefined state (using \Macro\let).\\
  \bottomrule
\end{tabularx}


\clearpage
\section{Temporary Variables}
The following variables are defined and used by the \LaTeX{} kernel commands as scratch registers and macros.
They can be used with care, but should be only redefined inside a local group.

\par\bigskip\noindent
\begin{tabularx}{\linewidth}{llX}
  \toprule
  Temp variable      & Type              & Note \\
  \midrule
  \Macro\count@      & counter           & \\
  \Macro\@tempcnta   & counter           & \\
  \Macro\@tempcntb   & counter           & \\
  \Macro\dimen@      & dimension         & \\
  \Macro\dimen@i     & dimension         & Marked as ``global only'' \\
  \Macro\dimen@ii    & dimension         & \\
  \Macro\@tempdima   & dimension \\
  \Macro\@tempdimb   & dimension \\
  \Macro\@tempdimc   & dimension \\
  \Macro\@tempa      & macro \\
  \Macro\@tempb      & macro \\
  \Macro\@tempc      & macro \\
  \Macro\@gtempa     & macro             & For temporary definitions which must be made global\\
  \Macro\skip@       & skip              & \\
  \Macro\@tempskipa  & skip \\
  \Macro\@tempskipb  & skip \\
  \Macro\toks@       & token register    & \\
  \Macro\@temptokena & token register \\
  \Macro\if@tempswa  & if switch         & Comes with the usual setters \Macro\@tempswatrue and \Macro\@tempswafalse \\
  \Macro\@tempboxa   & box register \\
  \Macro\@let@token  & `let'  & Used by \Macro\@ifnextchar to temporary store the next token using \Macro\futurelet. Can be used for similar purposes. \\
  \bottomrule
\end{tabularx}

\clearpage
\section{Macros}

\subsection{Macro Definition}
\par\bigskip\noindent
\begin{tabularx}{\linewidth}{lX}
   \toprule
   Macro & Description \\
   \midrule
   \Macro\@namedef{<name>}<parameter list>{<definition>} & Define macro \MacroArgs!\@backslashchar!<name> \relax(using \Macro\def). \\
   \Macro\@nameuse{<name>}  &  Expands to \MacroArgs!\@backslashchar!<name>. \\
   \Macro\@ifnextchar<token>{<yes>}{<no>} & Tests if next non-space token is equal to \MacroArgs<token>. \\
   \Macro\@ifstar{<yes>}{<no>} & Tests of next non-space token is `|*|`. Removes star for \MacroArgs<yes> branch.   \\
   \Macro\@dblarg{<cmd>}{<arg>} & Expands to \MacroArgs<cmd>[<arg>]{<arg>}. \\
   \Macro\@ifundefined{<name>}{<yes>}{<no>} & Tests if \MacroArgs!\@backslashchar!<name> is defined (and not equal to \Macro\relax). \\
   \Macro\@ifdefineable!\@backslashchar!<name>{<yes>} & Tests if \MacroArgs!\@backslashchar!<name> is undefined, \MacroArgs<name> not `|relax|'
        and doesn't start with `|end|', and if \Macro{end}<name> is not defined. \\
   \Macro\@onlypreamlbe<macro> & The given \meta{macro} is marked as only be valid in the preamble. It will be redefined as an error message AtBeginDocument.\\
   \Macro\@star@or@long & Tests for a following `|*|', if found \Macro\l@ngrel@x will be let to \Macro\relax, but \Macro\long otherwise. \\
   \Macro\@testopt{<1>}{<2>}  &  Short for \Macro\@ifnextchar'['{<1>}{<1>[{<2>}]}. \\
   \Macro\@protected@testopt & Robust version of \Macro\@testopt. ???? \\
   \bottomrule
\end{tabularx}

\subsection{Expanding/Gobbling Arguments}
\par\bigskip\noindent
\begin{tabularx}{\linewidth}{lX}
   \toprule
   Macro & Description \\
   \midrule
   \Macro\@gobble{<arg>} & Removes (gobbles) argument. (long)\\
   \Macro\@gobbletwo{<arg 1>}{<arg 2>} & Removes (gobbles) two arguments. (long)\\
   \Macro\@gobblefour{<1>}{<2>}{<3>}{<4>} & Removes (gobbles) four arguments. (long)\\
   \Macro\@firstofone{<arg>} & Expands to \meta{arg}, i.e. is used to remove braces. (long)\\
   \Macro\@iden{<arg>} & Identity. Same as Macro\@firstofone for compatibility reasons. (long)\\
   \Macro\@firstoftwo{<1>}{<2>}  & Expands to \meta{1}, discards \meta{2}. (long)\\
   \Macro\@secondoftwo{<1>}{<2>} & Expands to \meta{2}, discards \meta{1}. (long)\\
   \Macro\@thirdofthree{<1>}{<2>}{<3>} & Expands to \meta{3}, discards \meta{1} and \meta{2}. (long)\\
   \Macro\@expandtowargs\AlsoMacro\macro{<1>}{<2>} & Expands the two arguments using \Macro\edef and feeds it to \Macro\macro.\\
   \bottomrule
\end{tabularx}

\subsection{Loops}
\par\bigskip\noindent
\begin{tabularx}{\linewidth}{lX}
   \toprule
   Macro & Description \\
   \midrule
   \Macro\loop' ... '\AlsoMacro\iterate' ... '\AlsoMacro\repeat & \\
   \Macro\@whilenum <test>  \AlsoMacro\do {<body>}  &  While loop with \Macro\ifnum test.  \\
   \Macro\@whiledim <test>  \AlsoMacro\do {<body>}  &  While loop with \Macro\ifdim test.  \\
   \Macro\@whilesw <switch> \AlsoMacro\fi {<body>}  &  While loop with \MacroArgs<switch> test.  \\
   \Macro\@for!\@backslashchar!<macro>':='<list>\AlsoMacro\do{<body>} & For loop. The \MacroArgs<list> is supposed to expand to a comma separated list.
        Defines \MacroArgs!\@backslashchar!<macro> to each element of the list and executes \meta{body} each time.
        Supports an empty lists without errors. \\
   \Macro\@tfor!\@backslashchar!<macro>':='<list>\AlsoMacro\do{<body>} & For loop. The \MacroArgs<list> is not expanded and taken as a list of tokens or |{...}|.
        Defines \MacroArgs!\@backslashchar!<macro> to each element of the list and executes \meta{body} each time.
        Supports an empty lists without errors. \\
   \Macro\@break@tfor & Break out of a \Macro\@tfor loop. This should be called \emph{inside} the scope of an |\fi|.\\
   \bottomrule
\end{tabularx}
NOTES:\\
1. These macros use no |\@temp| sequences.\\
2. These macros do not work if the body contains anything that
looks syntactically to TeX like an improperly balanced |\if \else \fi|.\\

\subsection{Auxiliary Macros}
This auxiliary macros were originally defined to handle font changes but can be used for other code as well.
\par\bigskip\noindent
\begin{tabularx}{\linewidth}{lX}
   \toprule
   Macro & Description \\
   \midrule
   \Macro\ifnot@nil & Takes one argument and checks if it is the token \Macro\@nil. If so calls \Macro\@gobble, otherwise \Macro\@firstofone on the following argument. \\
   \Macro\@nil  & This macro is undefined on purpose and used as endmarker in loops and other macros which process token lists. \\
   \Macro\@nnil & Definition contains only \Macro\@nil and is used beside others to test for the presents of \Macro\@nil in \Macro\ifnot@nil. Is also used as endmarker.\\
   \Macro\remove@to@nnil & Removes everything behind it until and including \Macro\@nnil.\\
   \Macro\remove@star & Removes everything behind it until and including `|*|'.\\
   \bottomrule
\end{tabularx}

\subsection{Messages}
   \DescribeMacro\MessageBreak
   \noindent
   Inside a message this macro create a new line followed by a continuation line begun with \Macro\@msg@continuation. Outside it is equal to |\relax|.

   \DescribeMacro\GenericInfo{<continuation>}{<message>}
   \noindent
   Prints \meta{message} to a log file. Included \Macro\MessageBreak\relax's will cause a new line which start with \meta{continuation}.

   \DescribeMacro\GenericError!\footnotesize!{<continuation>}{<error message>}{<where to go for further information>}{<help text>}
   \noindent
   Print error message to log file followed by the `further information' line.
   The help text is displayed if the user presses `|h|'.

\par\bigskip\noindent
\begin{tabularx}{\linewidth}{lX}
  \toprule
  Name   &   Description    \\
  \midrule
   \Macro\strip@prefix & removes everything up to and including to the next |>| \\
   \Macro\@sanitize & Changes catcodes of everything except braces to `other' (12).\\
   \Macro\@onelevel@sanitize!\@backslashchar!<macro> & Sanitizes \meta{\textbackslash macro}, turns it definition into verbatim code. Resulting characters except spaces are in catcode `other' (12)!
                              Uses \Macro\meaning and \Macro\strip@prefix. \\
   \Macro\filename@parse{<filename>} & Parses \meta{filename} and provides its directory, base and extension
                                in \Macro\filename@area, \Macro\filename@base and \Macro\filename@ext. The latter is let to \Macro\relax if it does not exists. \\
   \Macro\wlog{<log message>}  & Write on log file only. \\
   \Macro\@settopoint{<register>} & Rounds register to whole number of points. \\
   \Macro\rem@pt<dimension value> & Awaits a value dimension value (iii.fffpt) as string where the `pt' is removed. If `fff' is numerical equal to 0, it and the decimal dot are removed as well. \\
   \Macro\strip@pt<dimension>     & Expands dimension using \cs{the} and strips the `pt' using \cs{rem@pt}. \\
   \Macro\addto@hook{<token register>}{<code>} & Appends code to the token register. \\
   \Macro\g@addto@hook{!\@backslashchar!<macro>}{<code>} & Appends \meta{code} to the definition of \MacroArgs!\@backslashchar!<macro>.\\
   \Macro\null & Empty \Macro\hbox. Good to fill places which must not be empty. \\
   \Macro\strutbox & Box with dimension of |\strut|. Can be used to extract this dimension with |\ht\strutbox| and |\dp\strutbox| (width=0). \\
   \Macro\@arstrutbox &! Defined inside array and tabular. Like \Macro\strutbox but stretched by \Macro\arraystretch. \\
 \bottomrule
\end{tabularx}
%   \Macro\void@box & (void) & newbox & permanently void box register \\

\section*{File related Macros}

\par\bigskip\noindent
\begin{tabularx}{\linewidth}{lX}
   \toprule
   Macro & Description \\
   \midrule
   \Macro\if@filesw & If false the package should not be produce or write to output files. Set false by \cs{nofiles}.\\
   \Macro\if@partsw & \\
   \Macro\input@path & List of input paths. Each path should be enclosed in braces with no delimiters between paths. \\
   \Macro\@filelist & The comma separated list of all files read so far. Only active if \cs{listfiles} is used in the preamble. \\
   \Macro\@currname & Name of the current package or option. \\
   \Macro\@currnext & Current file extension. \\
   \Macro\@classoptionslist & List of options of the main class. \\
   \Macro\@unusedoptionlist & List of options to the main class that haven't been declared. \\
\end{tabularx}
\par\noindent
\begin{tabularx}{\linewidth}{lX}
   \midrule
   \Macro\@iffileonpath{<filename>} & Check if given file is found by \TeX{} directly or in any of the directories given by \cs{input@path}. \\
   \Macro\@obsoletefile{<new>}{<obsolete>} & Prints warning message (only) that now a different file is used a input.\\
   \Macro\@addtofilelist{<filename>} &  Adds the given filename to the list of files. Only active if \cs{listfiles} is used in the preamble. \\
   \Macro\@ptionlist{<filename>} & Expands the option list of package, class or file given by full filename. \\
   \Macro\zap@space<text>!\verb*+ +!\AlsoMacro\@empty & Removes all spaces from \meta{text}. Expandable. \\
\end{tabularx}
\par\noindent
\begin{tabularx}{\linewidth}{lX}
   \midrule
   \Macro\@ifpackageloaded{<package>}{<true>}{<false>} & Tests if given package has been loaded. \\
   \Macro\@ifclassloaded{<class>}{<true>}{<false>} & Tests if given class has been loaded. \\
\end{tabularx}
\par\noindent
\begin{tabularx}{\linewidth}{lX}
   \midrule
   \Macro\@ifpackagelater{<package>}{<version>}{<true>}{<false>} & Tests if given package has been loaded with a version more recent than \meta{version}. \\
   \Macro\@ifclasslater{<class>}{<version>}{<true>}{<false>} & Tests if given class has been loaded with a version more recent than \meta{version}. \\
   \bottomrule
\end{tabularx}

\section{Other}

\subsection{Saved plain\TeX{} primitives}
The following plain\TeX{} macros are redefined by \LaTeX{} and therefore saved away first:

\par\bigskip\noindent
\begin{tabular}{lll}
   \toprule
   \LaTeX{} Macro & plain\TeX{} original & Description/Note \\
   \midrule
   \Macro\@@par   & \Macro{par}   & Some \LaTeX{} environments redefine \Macro{par} locally \\
   \Macro\@@input & \Macro\input  & Syntax: \Macro\input~<filename> \\
   \Macro\@@end   & \Macro\end    &  \\
   \bottomrule
\end{tabular}

\subsection*{Paragraph}
\begin{tabularx}{\linewidth}{lX}
   \toprule
   Macro & Description \\
   \midrule
   \Macro\@@par          & Plain\TeX{} primitive \cs{par}. \\
   \Macro\@setpar{<val>} & Used to make environment-wide changes to \cs{par}. Sets both \cs{par} and \cs{@par} to \meta{val}.  \\
   \Macro\@restorepar    & Defines \cs{par} to \cs{@par}. \\
   \bottomrule
\end{tabularx}

\subsection{Not yet sorted}
\par\bigskip\noindent
\begin{tabularx}{\linewidth}{lX}
   \toprule
   Macro & Description \\
   \midrule
   \Macro\@bsphack & \\
   \Macro\@esphack & Both of these macro ensure that the code between them does not insert any spaces into the document.
    The code itself should not produce any text and not change the mode.\\
   \Macro\@Esphack & Variant of \cs{@esphack} which sets the |@ignore| switch to true which causes an \cs{ignorespaces}
                after the \cs{end} of the environment.\\
   \Macro\@vbsphack & Variant of \cs{@bsphack} which ensure the invisible material is \emph{not} set in vmode. Not used by \LaTeX{} itself at the moment. \\
   \bottomrule
\end{tabularx}

\par\bigskip\noindent
\begin{tabularx}{\linewidth}{lX}
   \toprule
   Macro & Description \\
   \midrule
   \Macro\@currenvir & Name of the current environment. \\
   \Macro\hexnumber@{<number>} & Returns a single digit hexadecimal number (0--9, A--F) from given \meta{number}, which must either be a numeric register or a number ending with a space! \\
   \Macro\@starttoc{<ext>} & Reads the file with the given extension (|\jobname.|\meta{ext}) and opens it for writing afterwards. The file is initially empty. Creates the output file handle \Macro\tf@<ext>.\\
   \Macro\@writefile{<ext>}{<code>} & Writes code using the output handle \Macro\tf@<ext> if it exists.\\
   \Macro\@makeother{<letter>} & Changes the catcode of the letter to `other' (12). Special letters must be escaped with a backslash. \\
   \Macro\@begin@tempboxa{<box>}{<content>} & Stores \meta{content} into \Macro\@tempboxa as \meta{box} (\cs{hbox} or \cs{vbox}) and stores its dimension into \Macro\width, \Macro\height, \Macro\depth and \Macro\totalheight. \\
   \Macro\@end@tempboxa  & Ends a \Macro\@begin@tempboxa environment. \\
   \bottomrule
\end{tabularx}



\par\bigskip\noindent
\begin{tabularx}{\linewidth}{lX}
   \toprule
   Macro & Description \\
   \midrule
     \Macro\@alph{<number>} & Expands to lower case letter corresponding to the given number (1=a, 2=b, \ldots). Expands to \Macro\@ctrerr if number is larger then 26.\\
     \Macro\@Alph{<number>} & Expands to upper case letter corresponding to the given number (1=A, 2=B, \ldots). Expands to \Macro\@ctrerr if number is larger then 26.\\
   \bottomrule
\end{tabularx}


\end{document}



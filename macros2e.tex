%&pdflatex
% $Id$
\documentclass[12pt,a4paper]{article}
\usepackage[margin=1in]{geometry}
\usepackage{ydoc}
\usepackage{hyperref}
\usepackage{xcolor}
\hypersetup{filecolor=macrodesc}

\usepackage{booktabs}
\usepackage{tabularx}

\usepackage{zref-user}
\usepackage{zref-abspos}
\usepackage{zref-xr}
\usepackage{graphicx}

% Relative position of `source2e`
\def\sourceIIe{source2e}

\makeatletter
% Bug-fix for missing declaration:
\@ifundefined{ZREF@xr@ignored}{\newcount\ZREF@xr@ignored}{}

% Redefine the format of macro names to link to 'source2e':
\def\PrintMacroName#1{%
    {\macrodescstyle{\strut
    \begingroup
    \escapechar\m@ne
    \edef\name{\string #1}%
    \edef\@tempa{\noexpand\exthref{\name}{\noexpand\texttt{\char92}\name}}%
    \expandafter\endgroup\@tempa
    }}%
}


\def\exthref#1#2{{%
    % Get page of 'desc-'iption label
    \edef\page{\zref@extract{desc-#1}{abspage}}%
    \ifnum0\page=0\relax
        % If it doesn't exist try the 'code-' label
        \edef\page{\zref@extract{code-#1}{abspage}}%
        \ifnum0\page=0\else
            %\Gin@defaultbp\x{\zposx{code-#1}sp}%
            \Gin@defaultbp\y{\zposy{code-#1}sp}%
            \zrefused{code-#1}%
            %\message{^^J USED LABEL^^Jcode-#1^^J}%
        \fi
    \else
        %\Gin@defaultbp\x{\zposx{desc-#1}sp}%
        \Gin@defaultbp\y{\zposy{desc-#1}sp}%
        \zrefused{desc-#1}%
        %\message{^^J USED LABEL^^Jdesc-#1^^J}%
    \fi
    \ifnum0\page=0\relax
        % Simply typeset the text if no label is found
        #2%
        %\message{^^J No reference for: '\string#2'^^J}%
    \else
        % Otherwise set link to Fit Horizontally at the given y-coordinate
        \href[{page=\page,pdfnewwindow=true,pdfremotestartview=FitH \y}]{file:\sourceIIe.pdf}%
            %{\llap{\tiny\page,\y\ }
            {#2}%
    \fi
}}

\usepackage{filecontents}
% The relevant part of the 'source2e' aux files.
% Included here to allow the compilation of 'macros2e' without compiling 'source2e':
\begin{filecontents}{source2e.aux}
\zref@newlabel{code-@addtofilelist}{\posx{4238270}\posy{34901368}\abspage{75}}
\zref@newlabel{code-addto@hook}{\posx{5476886}\posy{33123877}\abspage{131}}
\zref@newlabel{code-@Alph}{\posx{7025156}\posy{16140091}\abspage{113}}
\zref@newlabel{code-@alph}{\posx{7025156}\posy{21128801}\abspage{113}}
\zref@newlabel{code-arraystretch}{\posx{4857578}\posy{37422995}\abspage{266}}
\zref@newlabel{code-@arstrutbox}{\posx{5167232}\posy{35037485}\abspage{266}}
\zref@newlabel{code-@backslashchar}{\posx{4238270}\posy{18977786}\abspage{37}}
\zref@newlabel{code-@begin@tempboxa}{\posx{3928616}\posy{9993266}\abspage{241}}
\zref@newlabel{code-bgroup}{\posx{6715502}\posy{15049007}\abspage{25}}
\zref@newlabel{code-@break@tfor}{\posx{5167232}\posy{16753061}\abspage{45}}
\zref@newlabel{code-@bsphack}{\posx{6096194}\posy{22257965}\abspage{60}}
\zref@newlabel{code-@cclvi}{\posx{6715502}\posy{43371343}\abspage{20}}
\zref@newlabel{code-@cclv}{\posx{7025156}\posy{45382660}\abspage{20}}
\zref@newlabel{code-@charlb}{\posx{6405848}\posy{29257209}\abspage{73}}
\zref@newlabel{code-@charrb}{\posx{6405848}\posy{28470777}\abspage{73}}
\zref@newlabel{code-@classoptionslist}{\posx{3309308}\posy{48210219}\abspage{375}}
\zref@newlabel{code-@clsextension}{\posx{4547924}\posy{34146195}\abspage{375}}
\zref@newlabel{code-count@}{\posx{6715502}\posy{34827769}\abspage{21}}
\zref@newlabel{code-@ctrerr}{\posx{6405848}\posy{14734431}\abspage{51}}
\zref@newlabel{code-@currenvir}{\posx{5476886}\posy{44409131}\abspage{211}}
\zref@newlabel{code-@currext}{\posx{6096194}\posy{36531705}\abspage{375}}
\zref@newlabel{code-@currname}{\posx{5786540}\posy{38917215}\abspage{375}}
\zref@newlabel{code-dimen@ii}{\posx{6096194}\posy{32468473}\abspage{21}}
\zref@newlabel{code-dimen@i}{\posx{6405848}\posy{33254905}\abspage{21}}
\zref@newlabel{code-dimen@}{\posx{6715502}\posy{34041337}\abspage{21}}
\zref@newlabel{code-egroup}{\posx{6715502}\posy{14262575}\abspage{25}}
\zref@newlabel{code-@empty}{\posx{6715502}\posy{38295707}\abspage{45}}
\zref@newlabel{code-empty}{\posx{7025156}\posy{19820027}\abspage{25}}
\zref@newlabel{code-end}{\posx{7644464}\posy{23608005}\abspage{213}}
\zref@newlabel{code-@end@tempboxa}{\posx{4547924}\posy{46458406}\abspage{242}}
\zref@newlabel{code-@Esphack}{\posx{6096194}\posy{12680632}\abspage{61}}
\zref@newlabel{code-@esphack}{\posx{6096194}\posy{20051344}\abspage{61}}
\zref@newlabel{code-@expandtwoargs}{\posx{4238270}\posy{22043481}\abspage{37}}
\zref@newlabel{code-@filelist}{\posx{5786540}\posy{37940901}\abspage{75}}
\zref@newlabel{code-@firstofone}{\posx{5167232}\posy{30519669}\abspage{37}}
\zref@newlabel{code-@firstoftwo}{\posx{5167232}\posy{29733237}\abspage{37}}
\zref@newlabel{code-@flushglue}{\posx{5476886}\posy{9262445}\abspage{42}}
\zref@newlabel{code-@for}{\posx{7334810}\posy{33758110}\abspage{45}}
\zref@newlabel{code-g@addto@macro}{\posx{4547924}\posy{25950831}\abspage{383}}
\zref@newlabel{code-GenericError}{\posx{4857578}\posy{18832350}\abspage{46}}
\zref@newlabel{code-GenericInfo}{\posx{5167232}\posy{34364382}\abspage{46}}
\zref@newlabel{code-@gobblefour}{\posx{5167232}\posy{32733396}\abspage{37}}
\zref@newlabel{code-@gobble}{\posx{6405848}\posy{34306260}\abspage{37}}
\zref@newlabel{code-@gobbletwo}{\posx{5476886}\posy{33519828}\abspage{37}}
\zref@newlabel{code-hexnumber@}{\posx{5476886}\posy{37593387}\abspage{178}}
\zref@newlabel{code-hideskip}{\posx{6096194}\posy{29805034}\abspage{22}}
\zref@newlabel{code-@iden}{\posx{7025156}\posy{26733078}\abspage{37}}
\zref@newlabel{code-@ifclasslater}{\posx{4547924}\posy{40581829}\abspage{376}}
\zref@newlabel{code-@ifclassloaded}{\posx{4238270}\posy{50503979}\abspage{376}}
\zref@newlabel{code-@iffileonpath}{\posx{4547924}\posy{9868297}\abspage{73}}
\zref@newlabel{code-if@filesw}{\posx{5786540}\posy{51290411}\abspage{70}}
\zref@newlabel{code-ifin@}{\posx{7025156}\posy{39042468}\abspage{157}}
\zref@newlabel{code-ifnot@nil}{\posx{5786540}\posy{36415564}\abspage{143}}
\zref@newlabel{code-@ifpackagelater}{\posx{3928616}\posy{41368261}\abspage{376}}
\zref@newlabel{code-@ifpackageloaded}{\posx{3618962}\posy{51290411}\abspage{376}}
\zref@newlabel{code-if@partsw}{\posx{5786540}\posy{50503979}\abspage{70}}
\zref@newlabel{code-if@tempswa}{\posx{5476886}\posy{22906960}\abspage{42}}
\zref@newlabel{code-in@}{\posx{7644464}\posy{39828900}\abspage{157}}
\zref@newlabel{code-iterate}{\posx{6405848}\posy{50503979}\abspage{26}}
\zref@newlabel{code-@ixpt}{\posx{7025156}\posy{21577710}\abspage{131}}
\zref@newlabel{code-l@ngrel@x}{\posx{5786540}\posy{9583303}\abspage{32}}
\zref@newlabel{code-loop}{\posx{7334810}\posy{51290411}\abspage{26}}
\zref@newlabel{code-@makeother}{\posx{5476886}\posy{37501637}\abspage{216}}
\zref@newlabel{code-maxdimen}{\posx{6096194}\posy{30591466}\abspage{22}}
\zref@newlabel{code-MessageBreak}{\posx{4857578}\posy{38820830}\abspage{46}}
\zref@newlabel{code-@Miii}{\posx{7025156}\posy{29333376}\abspage{42}}
\zref@newlabel{code-@Mii}{\posx{7334810}\posy{30119808}\abspage{42}}
\zref@newlabel{code-@Mi}{\posx{7644464}\posy{30906240}\abspage{42}}
\zref@newlabel{code-@miv}{\posx{7334810}\posy{28546944}\abspage{42}}
\zref@newlabel{code-@MM}{\posx{7644464}\posy{41012047}\abspage{20}}
\zref@newlabel{code-m@ne}{\posx{7334810}\posy{39598789}\abspage{21}}
\zref@newlabel{code-@M}{\posx{7954118}\posy{41798479}\abspage{20}}
\zref@newlabel{code-@m}{\posx{7954118}\posy{42584911}\abspage{20}}
\zref@newlabel{code-@ne}{\posx{7644464}\posy{48528388}\abspage{20}}
\zref@newlabel{code-@nnil}{\posx{7025156}\posy{40564505}\abspage{45}}
\zref@newlabel{code-null}{\posx{7334810}\posy{17434517}\abspage{25}}
\zref@newlabel{code-@obsoletefile}{\posx{4547924}\posy{44898377}\abspage{75}}
\zref@newlabel{code-@onelevel@sanitize}{\posx{2999654}\posy{35705951}\abspage{41}}
\zref@newlabel{code-@onlypreamble}{\posx{4547924}\posy{22297287}\abspage{32}}
\zref@newlabel{code-@pkgextension}{\posx{4547924}\posy{33359763}\abspage{375}}
\zref@newlabel{code-p@}{\posx{7954118}\posy{27536597}\abspage{22}}
\zref@newlabel{code-@protected@testopt}{\posx{2999654}\posy{8960348}\abspage{33}}
\zref@newlabel{code-@ptionlist}{\posx{5476886}\posy{11641133}\abspage{375}}
\zref@newlabel{code-remove@star}{\posx{5167232}\posy{28813388}\abspage{143}}
\zref@newlabel{code-remove@to@nnil}{\posx{4238270}\posy{30386252}\abspage{143}}
\zref@newlabel{code-rem@pt}{\posx{6715502}\posy{26450904}\abspage{122}}
\zref@newlabel{code-repeat}{\posx{6715502}\posy{49717547}\abspage{26}}
\zref@newlabel{code-@sanitize}{\posx{5786540}\posy{41210975}\abspage{41}}
\zref@newlabel{code-@secondoftwo}{\posx{4857578}\posy{28946805}\abspage{37}}
\zref@newlabel{code-@settopoint}{\posx{5167232}\posy{25440291}\abspage{114}}
\zref@newlabel{code-sixt@@n}{\posx{6405848}\posy{46169092}\abspage{20}}
\zref@newlabel{code-skip@}{\posx{7025156}\posy{31682041}\abspage{21}}
\zref@newlabel{code-space}{\posx{7025156}\posy{22205537}\abspage{25}}
\zref@newlabel{code-@spaces}{\posx{6405848}\posy{15350469}\abspage{50}}
\zref@newlabel{code-@sptoken}{\posx{6096194}\posy{14075249}\abspage{40}}
\zref@newlabel{code-@star@or@long}{\posx{4547924}\posy{14826183}\abspage{32}}
\zref@newlabel{code-@starttoc}{\posx{5786540}\posy{18641297}\abspage{296}}
\zref@newlabel{code-strip@prefix}{\posx{4857578}\posy{23450605}\abspage{11}}
\zref@newlabel{code-strip@pt}{\posx{6096194}\posy{27237336}\abspage{122}}
\zref@newlabel{code-strutbox}{\posx{6096194}\posy{48302198}\abspage{27}}
\zref@newlabel{code-@tempboxa}{\posx{5786540}\posy{16872437}\abspage{42}}
\zref@newlabel{code-@tempcnta}{\posx{5786540}\posy{25924221}\abspage{42}}
\zref@newlabel{code-@tempcntb}{\posx{5786540}\posy{25137789}\abspage{42}}
\zref@newlabel{code-@tempdima}{\posx{5786540}\posy{20610595}\abspage{42}}
\zref@newlabel{code-@tempdimb}{\posx{5786540}\posy{19824163}\abspage{42}}
\zref@newlabel{code-@tempdimc}{\posx{5786540}\posy{19037731}\abspage{42}}
\zref@newlabel{code-@tempskipa}{\posx{5476886}\posy{14576072}\abspage{42}}
\zref@newlabel{code-@tempskipb}{\posx{5476886}\posy{13789640}\abspage{42}}
\zref@newlabel{code-@temptokena}{\posx{5167232}\posy{11558810}\abspage{42}}
\zref@newlabel{code-@testopt}{\posx{6096194}\posy{16635766}\abspage{33}}
\zref@newlabel{code-@tfor}{\posx{7025156}\posy{22626339}\abspage{45}}
\zref@newlabel{code-@thirdofthree}{\posx{4547924}\posy{24388280}\abspage{37}}
\zref@newlabel{code-thr@@}{\posx{7025156}\posy{46955524}\abspage{20}}
\zref@newlabel{code-toks@}{\posx{7025156}\posy{30895609}\abspage{21}}
\zref@newlabel{code-tw@}{\posx{7644464}\posy{47741956}\abspage{20}}
\zref@newlabel{code-@unusedoptionlist}{\posx{3309308}\posy{45103813}\abspage{375}}
\zref@newlabel{code-@vbsphack}{\posx{5786540}\posy{48472363}\abspage{62}}
\zref@newlabel{code-@viiipt}{\posx{6405848}\posy{23886943}\abspage{131}}
\zref@newlabel{code-@viipt}{\posx{6715502}\posy{26196177}\abspage{131}}
\zref@newlabel{code-@vipt}{\posx{7025156}\posy{28505410}\abspage{131}}
\zref@newlabel{code-voidb@x}{\posx{6405848}\posy{25177301}\abspage{22}}
\zref@newlabel{code-@vpt}{\posx{7334810}\posy{30814644}\abspage{131}}
\zref@newlabel{code-@whiledim}{\posx{5786540}\posy{33569477}\abspage{44}}
\zref@newlabel{code-@whilenum}{\posx{5786540}\posy{37331243}\abspage{44}}
\zref@newlabel{code-@whilesw}{\posx{6096194}\posy{28955743}\abspage{44}}
\zref@newlabel{code-wlog}{\posx{7334810}\posy{37213279}\abspage{21}}
\zref@newlabel{code-@writefile}{\posx{5476886}\posy{29581431}\abspage{212}}
\zref@newlabel{code-@xiipt}{\posx{6715502}\posy{14650009}\abspage{131}}
\zref@newlabel{code-@xipt}{\posx{7025156}\posy{16959243}\abspage{131}}
\zref@newlabel{code-@xivpt}{\posx{6715502}\posy{12340776}\abspage{131}}
\zref@newlabel{code-@xpt}{\posx{7334810}\posy{19268476}\abspage{131}}
\zref@newlabel{code-@xviipt}{\posx{6405848}\posy{10031542}\abspage{131}}
\zref@newlabel{code-@xxpt}{\posx{7025156}\posy{7722309}\abspage{131}}
\zref@newlabel{code-@xxvpt}{\posx{6715502}\posy{51290411}\abspage{132}}
\zref@newlabel{code-@xxxii}{\posx{6715502}\posy{33202606}\abspage{42}}
\zref@newlabel{code-zap@space}{\posx{5786540}\posy{42049835}\abspage{381}}
\zref@newlabel{code-z@}{\posx{7954118}\posy{26750165}\abspage{22}}
\zref@newlabel{code-z@skip}{\posx{6715502}\posy{25963733}\abspage{22}}
\zref@newlabel{desc-@dblarg}{\posx{9520414}\posy{25170955}\abspage{31}}
\zref@newlabel{desc-filename@parse}{\posx{9520414}\posy{18058841}\abspage{12}}
\zref@newlabel{desc-@ifdefinable}{\posx{9520414}\posy{20488772}\abspage{31}}
\zref@newlabel{desc-@ifnextchar}{\posx{9520414}\posy{30675979}\abspage{31}}
\zref@newlabel{desc-@ifstar}{\posx{9520414}\posy{27694091}\abspage{31}}
\zref@newlabel{desc-@ifundefined}{\posx{9520414}\posy{22848068}\abspage{31}}
\zref@newlabel{desc-input@path}{\posx{9520414}\posy{30659958}\abspage{12}}
\zref@newlabel{desc-input}{\posx{9520414}\posy{26334582}\abspage{374}}
\zref@newlabel{desc-@namedef}{\posx{9520414}\posy{33858116}\abspage{31}}
\zref@newlabel{desc-@nameuse}{\posx{9520414}\posy{32248843}\abspage{31}}
\zref@newlabel{desc-@@par}{\posx{9520414}\posy{25506176}\abspage{54}}
\zref@newlabel{desc-@restorepar}{\posx{9520414}\posy{27060836}\abspage{54}}
\zref@newlabel{desc-@setpar}{\posx{9520414}\posy{30606086}\abspage{54}}
\end{filecontents}

% Load the external references
\zexternaldocument{\sourceIIe}

\title{List of internal \LaTeX2e Macros useful to Package Authors}
\author{Compiled by Martin Scharrer\\\url{martin@scharrer-online.de}}
\date{Version 0.2 -- February 9th 2011}

\makeatletter
\begin{document}
\maketitle
\begin{abstract}
This document lists the internal macros defined by the \LaTeX2e base files which can be also useful
to package authors.  A more detailed description can be found in the `source2e' document from which
this list was compiled.

This document is not yet complete in content and format and may miss some macros. Several macros are
not yet sorted in the correct category.
\end{abstract}

\tableofcontents

\clearpage

\section{Constants}

\subsection{Number Constants}

Some of the following integer values are defined using \Macro\countdef\relax (for $-1$),
\Macro\chardef\relax (for values between 0--255) and \Macro\mathchardef\relax ($>255$).  They are
robust and do not expand in an \Macro\edef\relax context. When used on the right side of an
assignment the act the same way as a count register.  The |\m@ne| is a real count register and can
still be modified, but doing so would certainly break various code.  In the terms defined by
\emph{The TeXbook}, this integer constants yield \emph{internal integers} rather than \emph{integer
denotations}.

The other numbers are defined as normal macros, which will simply expand to the containing number.
They were defined
to be used to set font sizes, which explains the non-integer numbers.  Please note that if they
are used for an assignment or other numeric context (e.g.\ |\ifnum|) TeX will keep expanding the
following tokens until a non-numeric token is found.%, e.g.\ a space or |\relax|.


\bigskip
\noindent\hbox to \linewidth\bgroup\hss
\begin{tabular}{lrl}
  \toprule
         Macro     & Value & Defined using \\
  \midrule
   \Macro\@ne      &     1 & chardef \\
   \Macro\tw@      &     2 & chardef \\
   \Macro\thr@@    &     3 & chardef \\
   \Macro\sixt@@n  &    16 & chardef \\
   \Macro\@xxxii   &    32 & chardef \\
   \Macro\@cclv    &   255 & chardef \\
   \Macro\@cclvi   &   256 & mathchardef \\
   \Macro\@m       &  1000 & mathchardef \\
   \Macro\@M       & 10000 & mathchardef \\
   \Macro\@Mi      & 10001 & mathchardef \\
   \Macro\@Mii     & 10002 & mathchardef \\
   \Macro\@Miii    & 10003 & mathchardef \\
   \Macro\@miv     & 10004 & mathchardef \\
   \Macro\@MM      & 20000 & mathchardef \\
  \bottomrule
\end{tabular}
\hss
\begin{tabular}{lrl}
  \toprule
         Macro     & Value & Defined using \\
  \midrule
   \Macro\@vpt     &     5 & def \\
   \Macro\@vipt    &     6 & def \\
   \Macro\@viipt   &     7 & def \\
   \Macro\@viiipt  &     8 & def \\
   \Macro\@ixpt    &     9 & def \\
   \Macro\@xpt     &    10 & def \\
   \Macro\@xipt    & 10.95 & def \\
   \Macro\@xiipt   &    12 & def \\
   \Macro\@xivpt   &  14.4 & def \\
   \Macro\@xviipt  & 17.28 & def \\
   \Macro\@xxpt    & 20.74 & def \\
   \Macro\@xxvpt   & 24.88 & def \\
  \\%\midrule
   \Macro\m@ne     &    -1 & countdef \\
  \bottomrule
\end{tabular}%
\hss\egroup


\subsection{Dimension Constants}
The following dimension and skip constants are defined using registers. They must not be changed.

\begin{center}
\begin{tabularx}{\linewidth}{lrlX}
  \toprule
    Macro           & Value         & Defined using & Notes \\
  \midrule
    \Macro\p@       & 1pt           & newdimen      & Can be used as a replacement of `|pt|' behind a number due to the resulting multiplication \\
    \Macro\z@       & 0pt           & newdimen      & Can be used both for 0pt and 0 \\
    \Macro\maxdimen & 16383.99999pt & newdimen      & Largest valid dimension \\
  \bottomrule
\end{tabularx}
\par\medskip
\begin{tabularx}{\linewidth}{lrlX}
  \toprule
    Macro             & Value                 & Defined using & Notes\\
    \midrule
    \Macro\z@skip     & 0pt plust0pt minus0pt & newskip       & \\
    \Macro\hideskip   & -1000pt plus 1fill    & newskip       & negative but can grow \\
    \Macro\@flushglue & 0pt plus 1fil         & newskip       & \\
  \bottomrule
\end{tabularx}
\end{center}

\subsection{String Constants}
The following macros hold common strings and are fully expandable.
Some special characters are defined here as their verbatim ASCII representation.
This makes them useful if such characters are to be written into an auxiliary file.

\bigskip\noindent
\begin{tabularx}{\linewidth}{llX}
  \toprule
    Macro                 & Value                        & Note\\
  \midrule
    \Macro\space          & One space                    & An explicit space (catcode 10 ``space'').\\
    \Macro\@spaces        & Four spaces                  & Contains $4\times$\Macro\space.\\
    \Macro\empty          & Empty string                 & Commonly used to define empty macros using \Macro\let. \\
    \Macro\@empty         & Empty string                 & Same as above. Used by the \LaTeX{} kernel commands. \\
    \Macro\@backslashchar & ``\texttt{\@backslashchar}'' & Backslash with catcode 12 (other), i.e. simple ASCII backlash usable in e.g.\ \Macro\write.\\
    \Macro\@percentchar   & ``\texttt{\@percentchar}''   & Percent character with catcode 12 (other), i.e. simple ASCII percent usable in e.g.\ \Macro\write.\\
    \Macro\@charlb        & ``\texttt{\@charlb}''        & Left brace with catcode 11 (letter). \\
    \Macro\@charrb        & ``\texttt{\@charrb}''        & Right brace with catcode 11 (letter). \\
    \Macro\@clsextension  & ``\texttt{cls}''             & Used for comparison with \Macro\@currext. \\
    \Macro\@pkgextension  & ``\texttt{sty}''             & Used for comparison with \Macro\@currext. \\
  \bottomrule
\end{tabularx}

\subsection{Token Constants}
The following macros are defined using \Macro\let<macro>'='<token> and therefore equal to this token. They are useful for |\ifx| comparisons
with tokens read by |\let| or |\futurelet|.

\bigskip\noindent
\begin{tabular}{llll}
 \toprule
   Name & Token & Catcode & Note \\
 \midrule
   \Macro\@sptoken & Space & 10 & Should not be confused with \Macro\space \\ 
   \Macro\bgroup   & |{|   &  1 & Begin of group \\
   \Macro\egroup   & |}|   &  2 & End of group \\
 \bottomrule
\end{tabular}

\subsection{Other}

\begin{tabularx}{\linewidth}{lllX}
  \toprule
    Macro            & Value   & Defined using & Notes \\
  \midrule
   \Macro\voidb@x    & (void)  & newbox        & permanently void box register \\
   \Macro\@undefined & (undef) & (undefined)   & This macro is not defined.
       It is used to test if other macros are undefined (using \Macro\ifx) or set them to an undefined state (using \Macro\let).\\
  \bottomrule
\end{tabularx}


\clearpage
\section{Variables}

\subsection{Temporary Variables}
The following variables are defined and used by the \LaTeX{} kernel commands as scratch registers and macros.
They can be used with care, but should only be redefined inside a local group to avoid interference with other
code. Care must also be taken if they are used together with external macros, which might use them as well
-- maybe not yet but in the next release.

\par\bigskip\noindent
\begin{tabularx}{\linewidth}{llX}
  \toprule
  Temp variable      & Type              & Note \\
  \midrule
  \Macro\count@      & counter           & \\
  \Macro\@tempcnta   & counter           & \\
  \Macro\@tempcntb   & counter           & \\
  \Macro\dimen@      & dimension         & \\
  \Macro\dimen@i     & dimension         & Marked as ``global only'' \\
  \Macro\dimen@ii    & dimension         & \\
  \Macro\@tempdima   & dimension \\
  \Macro\@tempdimb   & dimension \\
  \Macro\@tempdimc   & dimension \\
  \Macro\@tempa      & macro \\
  \Macro\@tempb      & macro \\
  \Macro\@tempc      & macro \\
  \Macro\@gtempa     & macro             & For temporary definitions which must be made global\\
  \Macro\skip@       & skip              & \\
  \Macro\@tempskipa  & skip \\
  \Macro\@tempskipb  & skip \\
  \Macro\toks@       & token register    & \\
  \Macro\@temptokena & token register \\
  \Macro\if@tempswa  & if switch         & Comes with the usual setters \Macro\@tempswatrue and \Macro\@tempswafalse \\
  \Macro\@tempboxa   & box register \\
  \Macro\@let@token  & `let'  & Used by \Macro\@ifnextchar to temporary store the next token using \Macro\futurelet. Can be used for similar purposes. \\
  \bottomrule
\end{tabularx}


\subsection{String Variables}

\par\bigskip\noindent
\begin{tabularx}{\linewidth}{lX}
   \toprule
   Macro & Description \\
   \midrule
   \Macro\@currext   & Extension of the current package or class file, empty outside. \\
   \Macro\@currname  & File name base of the current package or class file, empty outside. \\
   \Macro\@currenvir & Name of the current environment. \\
   \bottomrule
\end{tabularx}


\clearpage
\section{Macros}

\subsection{Macro Definition}
\DescribeMacro\@namedef{<name>}<parameter list>{<definition>}
\noindent
Defines macro \MacroArgs'\'<name>\relax using \Macro\def. Can be prefixed with \Macro\long and \Macro\global.

\DescribeMacro\@nameuse{<name>}
\noindent
Expands to macro \MacroArgs'\'<name>.

\DescribeMacro\@ifnextchar<token>{<yes>}{<no>}
\noindent
Tests if next non-space token is equal to \MacroArgs<token>. 

\DescribeMacro\@ifstar{<yes>}{<no>}
\noindent
Tests of next non-space token is `|*|`. Removes star for \MacroArgs<yes> branch.   

\DescribeMacro\@dblarg{<cmd>}{<arg>}
\noindent
Expands to \MacroArgs<cmd>[<arg>]{<arg>}. 

\DescribeMacro\@ifundefined{<name>}{<yes>}{<no>}
\noindent
Tests if \MacroArgs!\@backslashchar!<name> is defined (and not equal to \Macro\relax). 

\DescribeMacro\@ifdefinable'\'<name>{<yes>}
\noindent
Tests if \MacroArgs'\'<name> is undefined, \MacroArgs<name> not `|relax|'
and doesn't start with `|end|', and if \Macro{end}<name> is not defined. 

\DescribeMacro\@onlypreamble<macro>
\noindent
The given \meta{macro} is marked as only be valid in the preamble. It will be redefined as an error message AtBeginDocument.

\DescribeMacro\@star@or@long
\noindent
Tests for a following `|*|', if found \Macro\l@ngrel@x will be let to \Macro\relax, but \Macro\long
otherwise. It can the be used before \Macro\def as an potential prefix.

\DescribeMacro\@testopt{<1>}{<2>} 
\noindent
Short for \Macro\@ifnextchar'['{<1>}{<1>[{<2>}]}, i.e.\ \meta{2} will be used as default argument if non was given. 

\DescribeMacro\@protected@testopt
\noindent
Robust version of \Macro\@testopt? 

\DescribeMacro\addto@hook<token register>{<code>}
\noindent
Appends code to the token register. 

\DescribeMacro\g@addto@macro<macro>{<code>}
\noindent
Appends \meta{code} to the definition of \MacroArgs<macro>.\\

\subsection{Expanding/Gobbling Arguments}
\par\bigskip\noindent
\begin{tabularx}{\linewidth}{lX}
   \toprule
   Macro & Description \\
   \midrule
   \Macro\@gobble{<arg>} & Removes (gobbles) its argument. (long)\\
   \Macro\@gobbletwo{<arg 1>}{<arg 2>} & Removes (gobbles) two arguments. (long)\\
   \Macro\@gobblefour{<1>}{<2>}{<3>}{<4>} & Removes (gobbles) four arguments. (long)\\
   \Macro\@firstofone{<arg>} & Expands to \meta{arg}, i.e. is used to remove braces. (long)\\
   \Macro\@iden{<arg>} & Identity. Same as \Macro\@firstofone for compatibility reasons. (long)\\
   \Macro\@firstoftwo{<1>}{<2>}  & Expands to \meta{1}, discards \meta{2}. (long)\\
   \Macro\@secondoftwo{<1>}{<2>} & Expands to \meta{2}, discards \meta{1}. (long)\\
   \Macro\@thirdofthree{<1>}{<2>}{<3>} & Expands to \meta{3}, discards \meta{1} and \meta{2}. (long)\\
   \Macro\@expandtwoargs\AlsoMacro\macro{<1>}{<2>} & Expands the two arguments using \Macro\edef and feeds it to \Macro\macro.\\
   \bottomrule
\end{tabularx}

\subsection{Loops}
\par\bigskip\noindent
\begin{tabularx}{\linewidth}{lX}
   \toprule
   Macro & Description \\
   \midrule
   \Macro\loop' ... '\AlsoMacro\iterate' ... '\AlsoMacro\repeat & \\
   \Macro\@whilenum <test>  \AlsoMacro\do {<body>}  &  While loop with \Macro\ifnum test.  \\
   \Macro\@whiledim <test>  \AlsoMacro\do {<body>}  &  While loop with \Macro\ifdim test.  \\
   \Macro\@whilesw <switch> \AlsoMacro\fi {<body>}  &  While loop with \MacroArgs<switch> test.  \\
   \Macro\@for<macro>':='<list>\AlsoMacro\do{<body>} & For loop. The \MacroArgs<list> is supposed to expand to a comma separated list.
        Defines \MacroArgs<macro> to each element of the list and executes \meta{body} each time.
        Supports an empty lists without errors. \\
   \Macro\@tfor<macro>':='<list>\AlsoMacro\do{<body>} & For loop. The \MacroArgs<list> is not expanded and taken as a list of tokens or |{...}|.
        Defines \MacroArgs<macro> to each element of the list and executes \meta{body} each time.
        Supports an empty lists without errors. \\
   \Macro\@break@tfor & Break out of a \Macro\@tfor loop. This should be called \emph{inside} the scope of an |\fi|.\\
   \bottomrule
\end{tabularx}
\par\bigskip\noindent
NOTES:\par\vspace{-1ex}
\begin{enumerate}\itemsep=0pt
 \item These macros use no |\@temp| sequences.
 \item These macros do not work if the body contains anything that
looks syntactically to TeX like an improperly balanced |\if \else \fi|.
\end{enumerate}

\subsection{Auxiliary Macros}

\DescribeMacro\in@{<1>}{<2>}
\noindent
Checks if first argument occurs in the second and sets the switch \Macro\ifin@ accordantly.
The arguments are not expanded. This must be done beforehand.

This auxiliary macros were originally defined to handle font changes but can be used for other code as well.
\par\bigskip\noindent
\begin{tabularx}{\linewidth}{lX}
   \toprule
   Macro & Description \\
   \midrule
   \Macro\ifnot@nil{<1>}{<2>} & Checks if \meta{1} is the token \Macro\@nil. If so gobbles \meta{2} 
   otherwise uses \Macro\@firstofone to remove the braces around \meta{2}.\\
   \Macro\@nil  & This macro is undefined on purpose and used as endmarker in loops and other macros which process token lists. \\
   \Macro\@nnil & Definition contains only \Macro\@nil and is used beside others to test for the presents of \Macro\@nil in \Macro\ifnot@nil. Is also used as endmarker.\\
   \Macro\remove@to@nnil & Removes everything behind it until and including \Macro\@nnil.\\
   \Macro\remove@star & Removes everything behind it until and including `|*|'.\\
   \bottomrule
\end{tabularx}

\subsection{Messages}
\DescribeMacro\MessageBreak
\noindent
Inside a message this macro create a new line followed by a continuation text. Outside it is equal to |\relax|.

\DescribeMacro\GenericInfo{<continuation>}{<message>}
\noindent
Prints \meta{message} to a log file. Included \Macro\MessageBreak\relax's will cause a new line which start with \meta{continuation}.

\DescribeMacros
  \hbox{\Macro\GenericError{<continuation>}{<error message>}}%
  \hbox{\phantom{\ttfamily\textbackslash GenericError}\MacroArgs{<where to go for further information>}{<help text>}}%
\endDescribeMacros
\noindent
Print error message to log file followed by the `further information' line.
The help text is displayed if the user presses `|h|'.

\DescribeMacro\wlog{<log message>}
\noindent
Write on log file only.

\par\bigskip\noindent
\begin{tabularx}{\linewidth}{lX}
  \toprule
  Name   &   Description    \\
  \midrule
   \Macro\zap@space<text>!\verb*+ +!\AlsoMacro\@empty & Removes all spaces from \meta{text}. Expandable. \\
   \Macro\strip@prefix & Removes everything up to and including to the next `|>|' \\
   \Macro\@makeother{<letter>} & Changes the catcode of the letter to `other' (12). Special letters must be escaped with a backslash. \\
   \Macro\@sanitize & Changes catcodes of everything except braces to `other' (12).\\
   \Macro\@onelevel@sanitize<macro> & Sanitizes \meta{macro}, turns it definition into verbatim code. Resulting characters except spaces are in catcode `other' (12)!
                              Uses \Macro\meaning and \Macro\strip@prefix. \\
 \bottomrule
\end{tabularx}

\par\bigskip\noindent
\begin{tabularx}{\linewidth}{lX}
  \toprule
  Name   &   Description    \\
  \midrule
   \Macro\@settopoint<register>   & Rounds register to whole number of points. \\
   \Macro\rem@pt<dimension value> & Awaits a value dimension value (\MacroArgs<int>'.'<frac>'pt') as string where the `|pt|' is removed. If \meta{frac} is numerical equal to 0, then it and the decimal dot are removed as well. \\
   \Macro\strip@pt<dimension>     & Expands dimension using \cs{the} and strips the `|pt|' using \Macro\rem@pt. \\
 \bottomrule
\end{tabularx}


\subsection{File related Macros}

\begin{tabularx}{\linewidth}{lX}
   \toprule
   Macro & Description \\
   \midrule
   \Macro\if@filesw & If false the package should not be produce or write to output files. Set to false by \cs{nofiles}.\\
   \Macro\if@partsw & \\
   \Macro\input@path & List of input paths. Each path should be enclosed in braces with no delimiters between paths. \\
   \Macro\@filelist & The comma separated list of all files read so far. Only active if \cs{listfiles} is used in the preamble. \\
   \Macro\@classoptionslist & List of options of the main class. \\
   \Macro\@unusedoptionlist & List of options to the main class that haven't been declared. \\
   \bottomrule
\end{tabularx}

\DescribeMacro\filename@parse{<filename>}
\noindent
Parses \meta{filename} and provides its directory, name base and extension in \Macro\filename@area,
\Macro\filename@base and \Macro\filename@ext. The latter is let to \Macro\relax if it does not
exists.

\DescribeMacro\@starttoc{<ext>}
\noindent
Reads the file with the given extension (\texttt{\string\jobname.}\meta{ext}) and opens it for
writing afterwards. The file is initially empty. Creates the output file handle \Macro\tf@<ext>.

\DescribeMacro\@writefile{<ext>}{<code>}
\noindent
Writes code using the output handle \Macro\tf@<ext> if it exists.


\DescribeMacro\@iffileonpath{<filename>}
\noindent
Check if given file is found by \TeX{} directly or in any of the directories given by \cs{input@path}. 

\DescribeMacro\@obsoletefile{<new>}{<obsolete>}
\noindent
Prints warning message (only) that now a different file is used a input.

\DescribeMacro\@addtofilelist{<filename>}
\noindent
Adds the given filename to the list of files. Only active if \cs{listfiles} is used in the preamble. 

\DescribeMacro\@ptionlist{<filename>}
\noindent
Expands the option list of package, class or file given by full filename. 

\DescribeMacro\@ifpackageloaded{<package>}{<true>}{<false>}
\noindent
Tests if given package has been loaded.

\DescribeMacro\@ifclassloaded{<class>}{<true>}{<false>}
\noindent
Tests if given class has been loaded.

\DescribeMacro\@ifpackagelater{<package>}{<version>}{<true>}{<false>}
\noindent
Tests if given package has been loaded with a version more recent than \meta{version}.

\DescribeMacro\@ifclasslater{<class>}{<version>}{<true>}{<false>}
\noindent
Tests if given class has been loaded with a version more recent than \meta{version}.


\subsection{Saved plain\TeX{} primitives}
The following plain\TeX{} macros are redefined by \LaTeX{} and therefore saved away first:

\par\bigskip\noindent
\begin{tabular}{lll}
   \toprule
   \LaTeX{} Macro & plain\TeX{} original & Description/Note \\
   \midrule
   \Macro\@@par   & \Macro{par}   & Some \LaTeX{} environments redefine \Macro{par} locally \\
   \Macro\@@input & \Macro\input  & Syntax: \Macro\input~<filename> \\
   \Macro\@@end   & \Macro\end    &  \\
   \bottomrule
\end{tabular}

\subsection{Paragraph}
\begin{tabularx}{\linewidth}{lX}
   \toprule
   Macro & Description \\
   \midrule
   \Macro\@@par          & Plain\TeX{} primitive \cs{par}. \\
   \Macro\@setpar{<val>} & Used to make environment-wide changes to \cs{par}. Sets both \cs{par} and \cs{@par} to \meta{val}.  \\
   \Macro\@restorepar    & Defines \cs{par} to \cs{@par}. \\
   \bottomrule
\end{tabularx}

\subsection{Space Hack}
\begin{tabularx}{\linewidth}{lX}
   \toprule
   Macro & Description \\
   \midrule
   \Macro\@bsphack & \\
   \Macro\@esphack & Both of these macro ensure that the code between them does not insert any spaces into the document.
    The code itself should not produce any text and not change the mode (e.g.\ start or stop math mode).\\
   \Macro\@Esphack & Variant of \cs{@esphack} which sets the |@ignore| switch to true which causes an \cs{ignorespaces}
                after the \cs{end} of the environment.\\
   \Macro\@vbsphack & Variant of \cs{@bsphack} which ensure the invisible material is \emph{not} set in vmode. Not used by \LaTeX{} itself at the moment. \\
   \bottomrule
\end{tabularx}

\subsection{Boxes}
\begin{tabularx}{\linewidth}{lX}
   \toprule
   Macro & Description \\
   \midrule
   \Macro\null & Empty \Macro\hbox. Good to fill places which must not be empty. \\
   \Macro\strutbox & Box with dimension of |\strut|, i.e.\ maximum height and depth of the current font and zero width. Can be used to extract this dimension
                     with \texttt{\string\ht\string\strutbox} and \texttt{\string\dp\string\strutbox}. \\
   \Macro\@arstrutbox & Defined inside array and tabular. Like \Macro\strutbox but stretched by \Macro\arraystretch. \\
   \Macro\@begin@tempboxa<box>{<content>} & Stores \meta{content} into \Macro\@tempboxa as \meta{box} (\cs{hbox} or \cs{vbox}) and stores its dimension into \Macro\width, \Macro\height, \Macro\depth and \Macro\totalheight. \\
   \Macro\@end@tempboxa  & Ends a \Macro\@begin@tempboxa environment. \\
   \bottomrule
\end{tabularx}

\subsection{Base Conversion}
\begin{tabularx}{\linewidth}{lX}
   \toprule
   Macro & Description \\
   \midrule
     \Macro\hexnumber@{<number>} & Returns a single digit hexadecimal number (\mbox{0--9}, \mbox{A--F}) from given \meta{number}, which must either be a numeric register or a number ending with a space! \\
     \Macro\@alph{<number>} & Expands to lower case letter corresponding to the given number (1=a, 2=b, \ldots). Expands to \Macro\@ctrerr if number is larger then~26.\\
     \Macro\@Alph{<number>} & Expands to upper case letter corresponding to the given number (1=A, 2=B, \ldots). Expands to \Macro\@ctrerr if number is larger then~26.\\
   \bottomrule
\end{tabularx}

\end{document}


